\documentclass[a4paper,10pt]{article}
\usepackage[utf8]{inputenc}
\usepackage[italian]{babel}
%opening
\title{Relazioni dal laboratorio di calcolo}
\author{Riccardo Iaconelli}

\begin{document}

\maketitle

\begin{abstract}
Durante il Laboratorio di calcolo del terzo anno vengono mostrate alcune tecniche numeriche per la soluzione di problemi comuni in Fisica. In particolare l'attenzione è posta su diverse tecniche per svolgere integrali, sia di tipo deterministico che Monte-Carlo. Il programma si divide in più parti: per cominciare una sezione in cui si investigano tecniche per ridurre l'imprecisione numerica implicita di un calcolatore, dopodiché si studiano imprecisioni e rapidità di diversi algoritmi per il calcolo di integrali, e infine si applicano queste tecniche a problemi fisici. Come ultimo esercizio si implementa il metodo di Runge-Kutta per risolvere numericamente equazioni differenziali.
\end{abstract}
\clearpage

\section{Serie numeriche e precisione floating point}
Questo primo esercizio viene eseguito per mostrare come si possono ridurre eventuali errori di calcolo introdotti dalle normali operazioni aritmetiche.

\subsection{Analisi teorica}
All'interno del calcolatore le cifre non sono rappresentabili con precisione infinita. Con ogni operazione è possibile ottenere errori dovuti a due tipi di problemi.
\begin{description}
 \item[Errore di overflow (o underflow)] Questo errore avviene quando il risultato è più grande del massimo numero rappresentabile, o più piccolo del minimo numero rappresentabile.
 \item[Errore di troncamento] Quando la precisione della \textit{mantissa} 
\end{description}

Per vedere come questo è possibile, studiamo come sono rappresentati i numeri all'interno di un calcolatore.

\section{Integrazione con metodi deterministici}
\subsection{Metodo dei Trapezi}
\subsection{Metodo di Simpson}
\subsection{Metodo delle Quadrature Gaussiane}
\subsection{Conclusioni}

\section{Integrazione Monte-Carlo}
\subsection{Ago di Buffon}

\section{Integali di Feynman}

\section{Calcolo di equazioni differenziali}

\end{document}
